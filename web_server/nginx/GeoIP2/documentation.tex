\documentclass[12pt,a4paper]{article}
\usepackage[spanish]{babel}
\usepackage[utf8]{inputenc}
\usepackage[T1]{fontenc}
\usepackage{geometry}
\usepackage{hyperref}
\usepackage{listings}
\usepackage{xcolor}

\geometry{margin=2.5cm}

\lstset{
  basicstyle=\ttfamily\small,
  breaklines=true,
  frame=single,
  backgroundcolor=\color{gray!10}
}

\title{Módulo GeoIP2}
\author{Andrés Ruslan Abadías Otal (Nisamov)}
\date{}

\begin{document}
\maketitle

\section{Introducción}

El módulo GeoIP2 es un complemento para NGINX que permite aplicar restricciones de acceso en función del país de origen de la dirección IP del cliente.  
Esta guía amplía la documentación base del servidor web NGINX.

\section{Instalación de dependencias}

Actualizar el sistema e instalar los paquetes necesarios:

\begin{lstlisting}[language=bash]
sudo apt update && sudo apt upgrade -y
sudo apt install libmaxminddb0 libmaxminddb-dev mmdb-bin \
nginx-module-geoip2 -y
\end{lstlisting}

\section{Verificación del módulo}

Comprobar la existencia del módulo GeoIP2:

\begin{lstlisting}[language=bash]
ls /usr/lib/nginx/modules/ | grep geoip
\end{lstlisting}

El comando debe devolver el fichero:

\begin{lstlisting}
ngx_http_geoip2_module.so
\end{lstlisting}

\section{Base de datos MaxMind}

Crear una cuenta en MaxMind, generar una licencia y descargar la base de datos
\texttt{GeoLite2-Country.mmdb}.

Reubicar la base de datos:

\begin{lstlisting}[language=bash]
sudo mkdir -p /etc/nginx/geoip
sudo cp GeoLite2-Country.mmdb /etc/nginx/geoip/
\end{lstlisting}

Asignar los permisos adecuados:

\begin{lstlisting}[language=bash]
sudo chmod 644 /etc/nginx/geoip/GeoLite2-Country.mmdb
\end{lstlisting}

\section{Carga del módulo GeoIP2}

Editar el fichero principal de configuración de NGINX:

\begin{lstlisting}[language=bash]
sudo nano /etc/nginx/nginx.conf
\end{lstlisting}

Agregar la siguiente línea antes del bloque \texttt{events \{\}}:

\begin{lstlisting}
load_module modules/ngx_http_geoip2_module.so;
\end{lstlisting}

\section{Configuración del filtrado por país}

Dentro del bloque \texttt{http \{\}} ya existente en \texttt{nginx.conf}, añadir el siguiente contenido:

\begin{lstlisting}
http {
    geoip2 /etc/nginx/geoip/GeoLite2-Country.mmdb {
        $geoip2_country_code source=$remote_addr country iso_code;
    }

    map $geoip2_country_code $allowed_country {
        default yes;
        US no;
        MX no;
        AR no;
        CO no;
        CL no;
        PE no;
        BR no;
        VE no;
        EC no;
        BO no;
        PY no;
        UY no;
        DO no;
        CU no;
        GT no;
        HN no;
        NI no;
        SV no;
        CR no;
        PA no;
    }

    include /etc/nginx/conf.d/*.conf;
    include /etc/nginx/sites-enabled/*;
}
\end{lstlisting}

Esta configuración permite definir una lista de países bloqueados mediante su código ISO.

\section{Configuración del sitio web}

Editar el fichero de configuración del sitio:

\begin{lstlisting}[language=bash]
sudo nano /etc/nginx/sites-available/mipagina.es
\end{lstlisting}

Dentro del bloque \texttt{server \{\}}, añadir la lógica de bloqueo:

\begin{lstlisting}
server {
    listen 80;
    server_name mipagina.duckdns.org www.mipagina.duckdns.org;

    if ($allowed_country = no) {
        return 403;
    }

    root /var/www/mipagina.es;
    index index.html index.htm;

    location / {
        try_files $uri $uri/ =404;
    }
}
\end{lstlisting}

\section{Recarga y comprobación}

Verificar la sintaxis de la configuración y recargar NGINX:

\begin{lstlisting}[language=bash]
sudo nginx -t
sudo systemctl reload nginx
\end{lstlisting}

Si no se producen errores, el filtrado por país quedará activo en el servidor web.

\end{document}
