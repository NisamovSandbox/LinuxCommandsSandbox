\documentclass[12pt,a4paper]{article}
\usepackage[spanish]{babel}
\usepackage[utf8]{inputenc}
\usepackage[T1]{fontenc}
\usepackage{geometry}
\usepackage{hyperref}
\usepackage{listings}
\usepackage{xcolor}

\geometry{margin=2.5cm}

\lstset{
  basicstyle=\ttfamily\small,
  breaklines=true,
  frame=single,
  backgroundcolor=\color{gray!10}
}

\title{Documentación NGINX}
\author{Andrés Ruslan Abadías Otal (Nisamov)}
\date{}

\begin{document}
\maketitle

\section{Instalación de NGINX}

Instalar los servicios necesarios para poder continuar:

\begin{lstlisting}[language=bash]
sudo apt install nginx
\end{lstlisting}

Posteriormente, acceder al fichero de configuración de red y establecer una IP fija en el equipo.

\section{Creación de la estructura web}

Acceder a la ruta \texttt{/var/www} y crear un directorio en su interior:

\begin{lstlisting}[language=bash]
# Acceder a la ruta
cd /var/www
# Crear un directorio en su interior
mkdir mipagina.es
# Otorgar permisos
sudo chmod 755 mipagina.es
# Acceder al directorio creado
cd mipagina.es
# Crear un index.html, que será la página principal
nano index.html
\end{lstlisting}

Tras crear y abrir el fichero \texttt{index.html}, agregar la estructura básica de una página web:

\begin{lstlisting}[language=HTML]
<!DOCTYPE html>
<html lang="es">
<head>
    <meta charset="UTF-8">
    <meta name="viewport" content="width=device-width, initial-scale=1.0">
    <title>Mi Página</title>
</head>
<body>
    <p>Contenido cuerpo de página</p>
</body>
</html>
\end{lstlisting}

\section{Configuración de DuckDNS}

Acceder a \url{https://www.duckdns.org} y registrarse.

Una vez registrado, crear un subdominio (por ejemplo \texttt{mipagina.duckdns.org}) y asociarlo a la IP pública del equipo.

DuckDNS se encarga de la resolución DNS automáticamente, por lo que no es necesario modificar el fichero \texttt{/etc/hosts}.  
Tras la propagación DNS, el dominio será accesible desde cualquier dispositivo con conexión a Internet.

\section{Configuración de NGINX}

Acceder a la ruta de NGINX \texttt{/etc/nginx/sites-available} y crear el fichero de configuración:

\begin{lstlisting}[language=bash]
# Acceder a la ruta
cd /etc/nginx/sites-available
# Copiar el fichero por defecto y renombrarlo
sudo cp default mipagina.es
# Editar el fichero
sudo nano mipagina.es
\end{lstlisting}

Dentro del fichero, agregar el siguiente contenido:

\begin{lstlisting}
server {
    listen 80;
    root /var/www/mipagina.es;
    index index.html index.htm;
    server_name mipagina.duckdns.org www.mipagina.duckdns.org;

    location / {
        try_files $uri $uri/ =404;
    }
}
\end{lstlisting}

Posteriormente, crear un enlace simbólico:

\begin{lstlisting}[language=bash]
sudo ln -s /etc/nginx/sites-available/mipagina.es /etc/nginx/sites-enabled/mipagina.es
\end{lstlisting}

Reiniciar el servicio NGINX:

\begin{lstlisting}[language=bash]
service nginx restart
service nginx status
\end{lstlisting}

Para comprobar su correcto funcionamiento, acceder a la página mediante el dominio configurado:

\begin{lstlisting}
mipagina.es.duckdns.org
www.mipagina.es.duckdns.org
\end{lstlisting}

\section{Configuración de HTTPS con OpenSSL}

Instalar el servicio OpenSSL:

\begin{lstlisting}[language=bash]
sudo apt install openssl
\end{lstlisting}

Acceder a la ruta \texttt{/etc/nginx} y crear un directorio para almacenar las claves:

\begin{lstlisting}[language=bash]
cd /etc/nginx
mkdir ssl
chmod 700 ssl
\end{lstlisting}

\subsection{Creación de claves y certificados}

\begin{lstlisting}[language=bash]
openssl genrsa -out /etc/nginx/ssl/clave_privada.key 4096
openssl req -new -key /etc/nginx/ssl/clave_privada.key -out /etc/nginx/ssl/solicitud.csr

cd ssl
ls -al

openssl x509 -req -days 365 \
-in /etc/nginx/ssl/solicitud.csr \
-signkey /etc/nginx/ssl/clave_privada.key \
-out /etc/nginx/ssl/certificado.crt

ls -al
\end{lstlisting}

\section{Configuración SSL en NGINX}

Editar el fichero habilitado:

\begin{lstlisting}[language=bash]
sudo nano /etc/nginx/sites-enabled/mipagina.es
\end{lstlisting}

Agregar el siguiente contenido:

\begin{lstlisting}
server {
    listen 80;
    root /var/www/mipagina.es;
    index index.html index.htm;
    server_name mipagina.duckdns.org www.mipagina.duckdns.org;

    location / {
        try_files $uri $uri/ =404;
    }
}

server {
    listen 443 ssl;
    server_name mipagina.duckdns.org www.mipagina.duckdns.org;
    root /var/www/mipagina.es;
    index index.html index.htm;

    ssl_certificate /etc/nginx/ssl/certificado.crt;
    ssl_certificate_key /etc/nginx/ssl/clave_privada.key;

    location / {
        try_files $uri $uri/ =404;
    }
}
\end{lstlisting}

Este contenido permite establecer conexión segura mediante HTTPS utilizando certificados SSL.

\section{Finalización}

Reiniciar los servicios para aplicar los cambios:

\begin{lstlisting}[language=bash]
service nginx restart
service nginx status
\end{lstlisting}

Si el procedimiento se ha realizado correctamente, la página estará accesible desde cualquier dispositivo con conexión a Internet.

\end{document}
