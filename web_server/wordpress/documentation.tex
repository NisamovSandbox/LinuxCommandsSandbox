\documentclass[12pt,a4paper]{article}
\usepackage[spanish]{babel}
\usepackage[utf8]{inputenc}
\usepackage[T1]{fontenc}
\usepackage{geometry}
\usepackage{hyperref}
\usepackage{listings}
\usepackage{xcolor}

\geometry{margin=2.5cm}

\lstset{
  basicstyle=\ttfamily\small,
  breaklines=true,
  frame=single,
  backgroundcolor=\color{gray!10}
}

\title{Documentación WordPress}
\author{Andrés Ruslan Abadías Otal (Nisamov)}
\date{}

\begin{document}
\maketitle

\section{Actualización del sistema}

Actualizar los paquetes de la máquina:

\begin{lstlisting}[language=bash]
sudo apt upgrade && sudo apt update -y
\end{lstlisting}

\section{Instalación de dependencias}

Instalar las dependencias y paquetería necesarias:

\begin{lstlisting}[language=bash]
sudo apt install apache2 php8.1 php8.1-bcmath php8.1-curl php8.1-gd \
php8.1-mbstring php8.1-mysql php8.1-pgsql php8.1-xml php8.1-zip \
mariadb-server mariadb-client wget
\end{lstlisting}

También es posible realizar la instalación de forma individual:

\begin{lstlisting}[language=bash]
sudo apt install apache2
sudo apt install php7.4
sudo apt install wget
sudo apt install mariadb-server
sudo apt install mariadb-client
sudo apt install php8.1
sudo apt install php8.1-mysql
sudo apt install php8.1-curl
sudo apt install php8.1-gd
sudo apt install php8.1-bcmath
sudo apt install php8.1-cgi
sudo apt install php8.1-ldap
sudo apt install php8.1-mbstring
sudo apt install php8.1-xml
sudo apt install php8.1-soap
sudo apt install php8.1-xsl
sudo apt install php8.1-zip
sudo apt install libapache2-mod-php php-mysql -y
\end{lstlisting}

Este método permite instalar todos los paquetes necesarios de forma centralizada, ahorrando tiempo y evitando instalaciones fragmentadas.

\section{Instalación de WordPress}

Mediante el entorno gráfico, descargar WordPress y descomprimir el contenido.  
Posteriormente, mover el directorio a la ruta raíz:

\begin{lstlisting}[language=bash]
sudo mv wordpress /wordpress
\end{lstlisting}

Otorgar los permisos necesarios:

\begin{lstlisting}[language=bash]
cd /wordpress
sudo chown www-data:www-data .
sudo chown www-data:www-data -R *
\end{lstlisting}

\section{Configuración de Apache}

Deshabilitar el sitio por defecto de Apache:

\begin{lstlisting}[language=bash]
cd /etc/apache2/sites-available
sudo a2dissite 000-default
\end{lstlisting}

Reiniciar el servicio para aplicar los cambios:

\begin{lstlisting}[language=bash]
sudo service apache2 restart
\end{lstlisting}

Crear un nuevo fichero de configuración para WordPress:

\begin{lstlisting}[language=bash]
sudo nano wordpress.conf
\end{lstlisting}

Contenido del fichero de configuración:

\begin{lstlisting}
<VirtualHost *:80>
  ServerAdmin webmaster@localhost
  DocumentRoot /wordpress

  <Directory /wordpress>
    DirectoryIndex index.php
    AllowOverride All
    Require all granted
  </Directory>

  ErrorLog ${APACHE_LOG_DIR}/error.log
  CustomLog ${APACHE_LOG_DIR}/access.log combined
</VirtualHost>
\end{lstlisting}

Habilitar el sitio:

\begin{lstlisting}[language=bash]
sudo a2ensite wordpress
\end{lstlisting}

Reiniciar Apache:

\begin{lstlisting}[language=bash]
sudo systemctl restart apache2
\end{lstlisting}

\section{Configuración de la base de datos}

Acceder al gestor de bases de datos:

\begin{lstlisting}[language=SQL]
sudo mysql -u root -p
\end{lstlisting}

Modificar el método de autenticación del usuario \texttt{root}:

\begin{lstlisting}[language=SQL]
update mysql.user set plugin='mysql_native_password' where user='root';
\end{lstlisting}

Actualizar los privilegios:

\begin{lstlisting}[language=SQL]
flush privileges;
\end{lstlisting}

Crear la base de datos para WordPress:

\begin{lstlisting}[language=SQL]
create database wordpress;
exit;
\end{lstlisting}

Ajustar los parámetros básicos de seguridad:

\begin{lstlisting}[language=bash]
sudo mysql_secure_installation
\end{lstlisting}

Durante el proceso se responderá afirmativamente a las preguntas de seguridad.  
En el último paso se solicitará una contraseña para MySQL, la cual debe guardarse de forma segura.

\section{Configuración de WordPress}

Crear o editar el fichero \texttt{wp-config.php} y añadir la siguiente configuración:

\begin{lstlisting}[language=PHP]
<?php
define('DB_NAME', 'wordpress');
define('DB_USER', 'root');
define('DB_PASSWORD', 'andres');
define('DB_HOST', 'localhost');
define('DB_CHARSET', 'utf8mb4');
define('DB_COLLATE', '');

define('AUTH_KEY',         ']&eKuM6^mw^;,|:[-p?_[XJBkMr<E8&cJGR(.k|1v%bl-Q7szq|ipsgb411e4U}G');
define('SECURE_AUTH_KEY',  '7|6f^<J>F?4RR%}%k(IY;s)cqY%M4cW*Yp?qXroB[(jf9Zwfzx|r$G{$J3r86qf<');
define('LOGGED_IN_KEY',    ')xhXtUA|AkftoP@R-dPm|iI{?!i7^T>s~/*(@{,naV#9i kZ1pCK]|(31K5}u*J7');
define('NONCE_KEY',        '4$pu@PCrf|dBg^4K_Q>Iqaf| 5St~GU,n<#nl`2PghIU G55z/N]lcl%l68kfHd-');
define('AUTH_SALT',        'MD.SxW+}g`q3Ub}LN>!|,P<+Ya5QFR(vd1H2kE&U(|cl-hGUPq<I${#Ahaxl*H4J');
define('SECURE_AUTH_SALT', 'E^p{[?C4}/0ZG:}7V)OBakc~cL]M=}X.=:12sq `XM|O+]3](54sZZamq9g&/woy');
define('LOGGED_IN_SALT',   '4@p34O)[p)X}N KVsWu_l5<oaXRH>U/{XR,?d8;vRyI=,(Z8vNy%}yG$uH8)ht|9');
define('NONCE_SALT',       'a2u!6q:UrqN96JM,tWy-W==Z&[:!&0pNrbiC{XQ0DdG%! ip^N<F1&M0 Gx|AsE(');

$table_prefix = 'wp_';
define('WP_DEBUG', false);

if (!defined('ABSPATH')) {
  define('ABSPATH', __DIR__ . '/');
}

require_once ABSPATH . 'wp-settings.php';
\end{lstlisting}

\section{Finalización}

Una vez completada la configuración, acceder desde el navegador a la ruta del servidor y completar el asistente de instalación de WordPress para finalizar el despliegue.

\end{document}
